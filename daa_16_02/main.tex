\title{\textsc{A Comparison of Time Complexity of Greedy, Dynamic Programming and Divide \& Conquer}}
\author{CS F364: Design \& Analysis of Algorithms
\\
Lec 16\_2, BITS Pilani}
\date{}

\documentclass[12pt]{article}

\begin{document}
\maketitle

\paragraph{}
Suppose for a given problem we have three algorithms by using the above algorithm design techniques: a greedy algorithm, a dynamic programming algorithm and a divide \& conquer algorithm. Divide and conquer algorithm will be least efficient because it is solving each sub-problem even if it is an overlapping sub-problem. Dynamic Programming algorithm will be more efficient than divide and conquer algorithm because it is avoiding the repeated solutions of sub-problems. If we compare the greedy algorithm with dynamic programming algorithm, the greedy algorithm will be more efficient than Dynamic Programming algorithm because greedy algorithm we do not solve any sub-problem for making a choice. In general, the time complexity will be:
\\
\\
Greedy $<$ Dynamic Programming $<$ Divide \& Conquer

\paragraph{Example:}
Consider the single source shortest path problem. Dijkstra's algorithm is greedy and has complexity $O(|E|log|V|)$. Bellman-Ford algorithm is dynamic programming algorithm and has complexity $O(|V||E|)$. Under the assumption of $|E| = O(|V|^2)$, we can design a divide and conquer algorithm by using the recurrence of bellman-ford algorithm of complexity $\Omega |V|^{|V|}$. 
\\
The Comparison of three algorithm using $|E| = O(|V|^2)$,
\\
\\
Greedy: $O(|V|^2 log|V|) < $ Dynamic Prog.: $O(|V|^3) < $  D\&C: $\Omega (|V|^{|V|})$.

\end{document}
This is never printed